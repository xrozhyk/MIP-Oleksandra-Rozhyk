% Metódy inžinierskej práce

\documentclass[10pt,twoside,slovak,a4paper]{article}

\usepackage[slovak]{babel}
%\usepackage[T1]{fontenc}
\usepackage[IL2]{fontenc} % lepšia sadzba písmena Ľ než v T1
\usepackage[utf8]{inputenc}
\usepackage{graphicx}
\usepackage{url} % príkaz \url na formátovanie URL
\usepackage{hyperref} % odkazy v texte budú aktívne (pri niektorých triedach dokumentov spôsobuje posun textu)

\usepackage{cite}
%\usepackage{times}

\pagestyle{headings}

\title{3D MODELOVANIE\thanks{Semestrálny projekt v predmete Metódy inžinierskej práce, ak. rok 2021/22, vedenie: Oleksandra Rozhyk}} % meno a priezvisko vyučujúceho na cvičeniach

\author{Oleksandra Rozhyk\\[2pt]
	{\small Slovenská technická univerzita v Bratislave}\\
	{\small Fakulta informatiky a informačných technológií}\\
	{\small \texttt{xrozhyk@stuba.sk}}
	}

\date{\small 18. oktober 2021} % upravte



\begin{document}

\maketitle

\begin{abstract}
\ldots
\end{abstract}



\section{Úvod}

Prečo 3d modelovanie?
Dlho som myslela o rámcovej téme Modelovanie v softvérovom inžinierstve. Nevedela som čo môžem napísať a čo by mohlo byť zaujímavé. A potom prišiel mi výborný nápad o 3d modelovaní pretože veľmi to ma zaujíma a myslím si že 3d modelovanie je veľmi populárne a s každým rokom stáva ešte viac populárnejším. Moja kamarátka študuje na niekom takom odbore a vždy mi ukazovala ako to robí a vyzeralo to veľmi realisticky a užasne.Vždy ma toto zaujímalo, čo to je, ako to funguje, a kde sa používa v našom živote. Na prví pohľad nemyslela som si že ono je také dôležité, populárne aj nikde som nerozmýšľala nad tým že každý deň sa s tým stretávame. Napríklad na telefóne, kedy hráme hry, na televízií, kedy pozeráme filmy (anime), na rôznych tabletoch aj keď ideme na ulice tak môžeme vidieť rôzne reklamy atď. To len je malá časť kde môžeme sa s tým stretnúť. Ešte veľmi ma zaujíma táto sféra v architektúre kedy kreslia ľudia plán budovy a rôznych iných vecí.Chcela by som napísať o tom čo ma samu zaujíma, o tom čo to je, v akých sférach sa používa, čím je to tak dôležité. Tiež by som vám chcela povedať viac o sekcií vývoja hier a filmov, to by bola taká veľká časť mojej práce.


\section{Nejaká časť} \label{nejaka}

Z obr.~\ref{f:rozhod} je všetko jasné.

\begin{figure*}[tbh]
\centering
%\includegraphics[scale=1.0]{diagram.pdf}
Aj text môže byť prezentovaný ako obrázok. Stane sa z neho označný plávajúci objekt. Po vytvorení diagramu zrušte znak \texttt{\%} pred príkazom \verb|\includegraphics| označte tento riadok ako komentár (tiež pomocou znaku \texttt{\%}).
\caption{Rozhodujúci argument.}
\label{f:rozhod}
\end{figure*}



\section{Iná časť} \label{ina}

Základným problémom je teda\ldots{} Najprv sa pozrieme na nejaké vysvetlenie (časť~\ref{ina:nejake}), a potom na ešte nejaké (časť~\ref{ina:nejake}).\footnote{Niekedy môžete potrebovať aj poznámku pod čiarou.}

Môže sa zdať, že problém vlastne nejestvuje\cite{Coplien:MPD}, ale bolo dokázané, že to tak nie je~\cite{Czarnecki:Staged, Czarnecki:Progress}. Napriek tomu, aj dnes na webe narazíme na všelijaké pochybné názory\cite{PLP-Framework}. Dôležité veci možno \emph{zdôrazniť kurzívou}.


\subsection{Nejaké vysvetlenie} \label{ina:nejake}

Niekedy treba uviesť zoznam:

\begin{itemize}
\item jedna vec
\item druhá vec
	\begin{itemize}
	\item x
	\item y
	\end{itemize}
\end{itemize}

Ten istý zoznam, len číslovaný:

\begin{enumerate}
\item jedna vec
\item druhá vec
	\begin{enumerate}
	\item x
	\item y
	\end{enumerate}
\end{enumerate}


\subsection{Ešte nejaké vysvetlenie} \label{ina:este}

\paragraph{Veľmi dôležitá poznámka.}
Niekedy je potrebné nadpisom označiť odsek. Text pokračuje hneď za nadpisom.



\section{Dôležitá časť} \label{dolezita}




\section{Ešte dôležitejšia časť} \label{dolezitejsia}




\section{Záver} \label{zaver} % prípadne iný variant názvu



%\acknowledgement{Ak niekomu chcete poďakovať\ldots}


% týmto sa generuje zoznam literatúry z obsahu súboru literatura.bib podľa toho, na čo sa v článku odkazujete
\bibliography{literatura}
\bibliographystyle{plain} % prípadne alpha, abbrv alebo hociktorý iný
\end{document}
