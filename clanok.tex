% Metódy inžinierskej práce

\documentclass[10pt,twoside,slovak,a4paper]{article}

\usepackage[slovak]{babel}
%\usepackage[T1]{fontenc}
\usepackage[IL2]{fontenc} % lepšia sadzba písmena Ľ než v T1
\usepackage[utf8]{inputenc}
\usepackage{graphicx}
\usepackage{url} % príkaz \url na formátovanie URL
\usepackage{hyperref} % odkazy v texte budú aktívne (pri niektorých triedach dokumentov spôsobuje posun textu)

\usepackage{cite}
%\usepackage{times}

\pagestyle{headings}

\title{3D MODELOVANIE}

\author{Oleksandra Rozhyk\\[2pt]
	{\small Slovenská technická univerzita v Bratislave}\\
	{\small Fakulta informatiky a informačných technológií}\\
	{\small \texttt{xrozhyk@stuba.sk}}
	}

\date{\small 18. oktober 2021} % upravte



\begin{document}

\maketitle

\begin{abstract}
\centering 3D MODELOVANIE
\end{abstract}



\section{Úvod}

Videohra, tiež označovaná ako počítačová alebo digitálna hra, je druh interaktív- neho digitálneho média zahŕňajúceho množstvo vzájomne prepojených aspek- tov, ako napríklad hernú logiku, modely postáv, animácie a mnohé ďalšie. Ich veľký počet prináša neustále možnosti na svoje čiastkové zlepšovanie, pričom sa cieli na najlepšie možné zvládnutie danej problematiky.
Riešiť 3D animácie reči vo videohrách je dôležité hlavne z dôvodu veľkého dopadu hier na človeka a spoločnosť, pričom reč je práve jedným z najdôležitej- ších prejavov ľudskej komunikácie. Tento problém sa neustále posúva a vývojári dosahujú stále lepšie výsledky, avšak neexistuje ustálený postup, ako riešiť túto problematiku.
Existujú programy ako FaceFX, ktoré dokážu model automaticky zanimovať, alebo program CrazyTalk8, ktorý ponúka aj analýzu zvukovej stopy a jej následnú animáciu, ale tieto programy sú komerčné a nad ich výsledkom nemáte význačnú kontrolu.
Táto práca sa zaoberá vytvorením animácie pier a tváre pri reči na 3D modele ľudskej hlavy. Jej výstupom je funkčný model so svojim systémom kostí (tzv. rig), ktorý ovláda pohyb. Charakter vznikol v kontexte videohier, čo znamená, že je ho možné použiť v aplikáciách, ktoré fungujú v reálnom čase, ako napríklad v počí- tačových hrách alebo vo virtuálnej realite.
Ide o výtvarnú prácu technického charakteru a jej obsah je určený primárne pre mierne pokročilých 3D grafikov alebo animátorov, ktorí sa zaujímajú o animáciu nielen tváre, ale aj modelu ako celku.


\section{Videohry}

\section{3D animácia} \label{ina}

\section{Typy modelov} \label{dolezita}

\section{Ešte dôležitejšia časť} \label{dolezitejsia}



\section{Záver} \label{zaver} % prípadne iný variant názvu



%\acknowledgement{Ak niekomu chcete poďakovať\ldots}


% týmto sa generuje zoznam literatúry z obsahu súboru literatura.bib podľa toho, na čo sa v článku odkazujete
\bibliography{literatura}
\bibliographystyle{plain} % prípadne alpha, abbrv alebo hociktorý iný
\end{document}
